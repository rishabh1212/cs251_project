\documentclass{beamer}
\usetheme{Pittsburgh}
\author[Abhinav \& Surya \& Rishabh]
{%
   \texorpdfstring{
        \begin{columns}
            \column{.4\linewidth}
            \centering
            Abhinav\\
            {140050054}\\
            {abhinavrondi11296@gmail.com}
            \column{.4\linewidth}
            \centering
            Surya\\
            {140050055}\\
            {suri892010@gmail.com}
            \column{.33\linewidth}
            \centering
            Rishabh\\
            {140050061}\\
            {rishabh6417@gmail.com}
        \end{columns}
   }
   {John Doe \& Jane Doe}
}
\title{Box2D project Car: A CS251 Report by Group 20.}
\date{\today}
\begin{document}
\begin{frame}
\titlepage
\end{frame}
\section{Introduction}
\begin{frame}
\frametitle{Introduction}
\begin{itemize}
\item The Box2D is a physics engine which helps us simulate the rube goldberg machine.
\item And one of them is simulation of a car in 2D space.
\item And there is more situations we can simulate with Box2D.
\end{itemize}
\end{frame}
\section{Examples}
\subsection{Technical Attributes}
\begin{frame}
\frametitle{Technical Attributes}
\begin{itemize}
\item This simulation always obeys physics laws and cannot be violated.
\item The simulation accuracy depends on the number of simuations per second and also on number of objects in dynamic condition.
\item It also depends on the number of functions called and how optimized the code is.
\end{itemize}
\end{frame}
\begin{frame}
\begin{itemize}
\frame{Efforts }
\item We have worked for a lot of weeks to develop the code and analyse it
\item We have worked hand in hand to reduce the burden from each other
\end{itemize}
\end{frame}
\begin{frame}
\begin{itemize}
\frame{Acknowledgements }
\item Google was always there whenever we wanted except when the lan was out
\item We also got a lot of support from Prof. Sharat Chandran and the TAs.
\item 
\end{itemize}
\end{frame}
\end{document}